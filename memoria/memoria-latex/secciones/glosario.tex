% GLOSARIO

% Si quieres incluir un glosario y una lista de abreviaturas en tu Trabajo Fin de Máster,
% sigue las instrucciones indicadas en la siguiente URL:
% https://www.overleaf.com/learn/latex/glossaries

\newacronym{acr_nlp}{NLP}{Procesamiento del Lenguaje Natural o \textit{Natural Language Processing}}
\newacronym{acr_squad}{SQuAD}{The Stanford Question Answering Dataset}
\newacronym{acr_ml}{ML}{Aprendizaje Automático o \textit{Machine Learning}}
\newacronym{acr_dl}{DL}{Aprendizaje Profundo o \textit{Deep Learning}}
\newacronym{acr_rnn}{RNN}{Redes Neuronales Recurrentes o \textit{Recurrent Neural Networks}}
\newacronym{acr_cnn}{CNN}{Redes Neuronales Convolucionales o \textit{Convolutional Neural Networks}}
\newacronym{acr_ffnn}{FFNN}{Redes Neuronales Prealimentadas o \textit{Feed Forward Neural Networks}}
\newacronym{acr_dnn}{DNN}{Redes Neuronales Profundas o \textit{Deep Neural Networks}}
\newacronym{acr_lstm}{LSTM}{Redes de Gran Memoria a Corto Plazo o \textit{Long Short-Term Memory}}
\newacronym{acr_tpu}{TPU}{Unidad de Procesamiento Tensorial o \textit{Tensor Processing Unit}}
\newacronym{acr_gpu}{GPU}{Unidad de Procesamiento Gráfico o \textit{Graphic Processing Unit}}
\newacronym{acr_html}{HTML}{Lenguaje de Marcas de Hipertexto o \textit{HyperText Markup Language}}

\newglossaryentry{gls_wordembeddings}
{
        name=Word Embeddings,
        text=\textit{word embeddings},
        description={es una técnica del procesamiento del lenguaje natural que consiste, básicamente, en asignar un vector a cada palabra. Este vector guarda información semántica, lo que permite que pueda ser asociado o disociado a otros vectores (palabras) según distintos contextos gramaticales}
}

\newglossaryentry{gls_transformer}
{
        name=Transformer,
        text=transformer,
        description={son una clase reciente de redes neuronales para datos de tipo secuenciales, basadas en la autoatención, que han demostrado estar bien adaptadas al texto y actualmente están impulsando importantes avances en el procesamiento del lenguaje natural}
}

\newglossaryentry{gls_softmax}
{
        name=Softmax,
        name=\textit{softmax},
        description={La función Softmax es una función de activación que transforma las salidas a una representación en forma de probabilidades, de tal manera que el sumatorio de todas las probabilidades de las salidas de 1. Devuelve la distribución de probabilidad de cada una de las clases soportadas en el modelo}
}

\newglossaryentry{gls_json}
{
        name={Archivo JSON},
        text={\textit{archivo .json}},
        description={Notación de Objeto Javascript o JavaScript Object Notation (JSON) es un formato de archivo sencillo para el intercambio de información. El formato JSON permite representar estructuras de datos (arrays) y objetos (arrays asociativos) en forma de texto}
}

\newglossaryentry{gls_tfidf}
{
        name={TF-IDF},
        description={TF-IDF (del inglés Term frequency – Inverse document frequency), frecuencia de término – frecuencia inversa de documento, es una medida numérica que expresa cuán relevante es una palabra para un documento en una colección, calculada a través de la frecuencia de ocurrencia del término en la colección de documentos}
}

\newglossaryentry{gls_computer_vision}
{
        name={Computer Vision},
        text={\textit{computer vision}},
        description={Disciplina científica que incluye métodos para adquirir, procesar, analizar y comprender las imágenes del mundo real con el fin de producir información numérica o simbólica para que puedan ser tratados por un ordenador.}
}

