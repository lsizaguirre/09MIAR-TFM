% OBJETIVOS

\cleardoublepage

\chapter{Objetivos}
\label{chapter-objetivos}

La publicación de BERT por parte de Google marcó el inicio de una nueva era en el machine learning obteniendo con este modelo excelentes resultados que definieron un nuevo estado del arte en muchas tareas y benchmarks en el campo del procesamiento de lenguaje natural.

El hecho de ser un modelo accesible, con una arquitectura unificada que permite un ajuste rápido y fácil, lo convierte en un modelo de referencia.

\section{Objetivo General}
\label{section-objetivo-general}

Basado en los fundamentos antes expuestos, el objetivo principal de este trabajo es: \textbf{Evaluar el uso y desempeño de BERT (Biderectional Encoder Representations for Transformers) como modelo en la resolución de tareas relacionadas al procesamiento del lenguaje natural, distintas a las que para en un principio fue preentrenado.} Específicamente, se trabajará en dos tareas, el análisis de sentimiento y el problema de respuesta a preguntas.
\medskip

\section{Objetivo Específicos}
\label{section-objetivos-especificos}

Para evaluar el uso y desempeño de este modelo en la resolución de problemas asociados al procesamiento de lenguaje natural, trabajaremos en los objetivos específicos citados a continuación:
\medskip

\begin{enumerate}[label=\destacado{\arabic*.}]
  \setlength\itemsep{1em}
  \item Identificar problemas típicos del procesamiento del lenguaje natural donde, el uso de modelos basados en \textit{transformers} hayan tenido un aporte considerable en su solución.

  \item Identificar y seleccionar un conjunto de datos para cada tarea que permita construir una solución a través del ajuste del modelo. Estudiar, limpiar y dividir los datos en los subconjuntos de entrenamiento, prueba y validación.

  \item Evaluar el estado del arte para cada uno de estos problemas, entendiendo el desempeño de distintos algoritmos empleados en su solución.
  
  \item Desarrollar soluciones basadas en el ajuste de BERT, adecuándose para resolver los problemas seleccionados.
  
  \item Comparar el desempeño de las soluciones desarrolladas con el estado del arte haciendo uso de las métricas apropiadas en cada benchmark.
\end{enumerate}